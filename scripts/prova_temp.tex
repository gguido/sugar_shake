\documentclass[a4paper,11pt]{article}
\usepackage[T1]{fontenc}
\usepackage[utf8]{inputenc}
\usepackage{lmodern}

\title{}
\author{}

\begin{document}

\maketitle
\tableofcontents

\begin{abstract}
\end{abstract}

\section{}

\begin{center}
\begin{equation}
ca = k*cb
\label{equ:control_1}
\end{equation}
\end{center}
La variazione nelle colonie controllo può essere modellizzata secondo l'equazione \ref{equ:control_1}, dove \textit{cb} è il tasso di infestazione delle colonie controllo prima del trattamento, \textit{ca} il tasso di infestazione delle colonie controllo dopo il trattamento, \textit{k} il coefficiente di variazione dell'infestazione dal momento del campionamento precedente il trattamento al momento del campionamento successivo al trattamento.
\begin{center}
\begin{equation}
ta = (1-eff)*k*tb
\label{equ:treat_1}
\end{equation}
\end{center}
Assumiamo quindi che, se le colonie del gruppo trattato non fossero state trattate, sarebbero andate incontro alla stessa variazione dell'infestazione \textit{k} di quelle di controllo. Essendo state però trattate la loro variazione da prima del trattamento (\textit{tb}) a dopo il trattamento (\textit{ta}) si può assumere equivalente a \begin{math}(1 - {eff})*k\end{math}, dove \textit{eff} è l'efficacia del trattamento.\\
Poiché, differentemente da \citet{henderson_tests_1955}.
, non ci troviamo in presenza di un'unica unità di controllo (un campo coltivato) ma di diversi alveari, assumiamo come \textit{k} la media delle variazioni delle infestazioni. Rsisolvendo rispetto a k si ottiene l'equazione \ref{eq:mean_growth}.
% corretto usare la media aritmetica?
\begin{center}
\begin{equation}
k = \frac{\sum_{i=1}^{n} \frac{ca_i}{cb_i}}{n}
\label{eq:mean_growth}
\end{equation}
\end{center}
Sostituendo nell'equazione \ref{equ:treat_1} il valore di \textit{k} trovato al punto \ref{eq:mean_growth} e risolvendo rispetto a \textit{eff} si ottiene l'equazione \ref{eq:ht_mod}.
\begin{center}
\begin{equation}
eff = 1-\frac{ta}{tb}*\frac{1}{\frac{\sum_{i=1}^{n} \frac{ca_i}{cb_i}}{n}}
\label{eq:ht_mod}
\end{equation}
\end{center}
L'equazione \ref{eq:ht_mod} così trovata è classica formula di \citet{henderson_tests_1955} (riportata al punto ) adattata al caso in esame.
\begin{center}
\begin{equation}
eff = 1-\frac{ta}{tb}*\frac{cb}{ca}
\label{eq:ht_mod}
\end{equation}
\end{center}

\end{document}